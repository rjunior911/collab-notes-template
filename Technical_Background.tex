\documentclass[./main.tex]{subfiles}
%  %% \usepackage{natbib}
%\usepackage[sorting=none,citestyle=authoryear]{biblatex}
\usepackage{geometry} %set the font and margins
\usepackage{bibentry} %allows for inline citations before the bibliography (say in a footnote or resume)
\usepackage[final]{graphicx} % Allows including images
\usepackage{amssymb}
\usepackage{amsmath}
\usepackage{amsthm} 
\usepackage{tabularx}
\usepackage{tikz} 
\usepackage{tikz-cd} %cd for Commutative Diagrams
% \usepackage{xypic}
\usepackage{caption}
\usepackage{subcaption}
\usepackage{wrapfig}
%% \usetikzlibrary{babel}

%had to redefine sections and subsections to put line breaks
% \usepackage[parfill]{parskip}    

% \makeatletter
% \def\section{\@startsection{section}{3}%
%   \z@{.5\linespacing\@plus.7\linespacing}{.1\linespacing}%
%   {\normalfont\itshape}}
% \def\subsection{\@startsection{subsection}{3}%
%   \z@{.5\linespacing\@plus.7\linespacing}{.1\linespacing}%
%   {\normalfont\itshape}}
% \def\subsubsection{\@startsection{subsubsection}{3}%
%   \z@{.5\linespacing\@plus.7\linespacing}{.1\linespacing}%
%   {\normalfont\itshape}}
% \makeatother

% ========================================
%TAKE 17 on lowercase math script fonts
% ========================================
% \let\mathbbalt\mathbb
% \usepackage{fontspec}
% \usepackage{unicode-math}
\usepackage[utf8]{inputenc}
% \usepackage{mathalfa} % more font options
\usepackage[cal=esstix,
% bb=ams,
% frak=ams,
scr=rsfs
]{mathalpha}
% \usepackage[cal=boondoxo]{mathalfa}
\usepackage{mathabx}

\usepackage{amsfonts} % more font options
% \usepackage{mathrsfs} %doesn't work, maybe need to fix texlive?
% \usepackage[mathscr]{eucal}
% \usepackage{boondox-cal}
% \usepackage{calligra}%breaks on Mac
% \DeclareMathAlphabet{\mathcal}{T1}{calligra}{m}{n}
% ========================================
% \let\mathbb\mathbbalt% UNIVERSAL RESET TO ORIGINAL \mathbb

\usepackage{pdfsync}

% \usepackage{ntheorem}
\usepackage{booktabs} % Allows the use of \toprule, \midrule and \bottomrule in tables

\newcolumntype{L}{>{\centering\arraybackslash}p{2cm}}
\newcolumntype{M}{>{\centering\arraybackslash}p{0.45t\textwidth}}
\newcolumntype{E}{>{\centering\arraybackslash}p{0.1\textwidth}}
\newcommand{\de}[1]{\textbf{#1}}

\newtheorem{thm}{Theorem}
\newtheorem{lem}{Lemma}
\newtheorem{conj}{Conjecture}
\newtheorem{conjecture}{Conjecture}
\newtheorem{prop}{Proposition}
\newtheorem*{definition}{Definition}
\newtheorem*{principle}{Principle}
\newtheorem*{project}{Project}
\newtheorem*{moral}{Moral}
\newtheorem*{question}{Question}
\newtheorem*{intuition}{Intuition}
\newtheorem*{fact}{Fact}
\newtheorem*{xample}{Example}

\renewcommand\qedsymbol{$\blacksquare$}

%Below lies the necessary commands to copy nlab diagrams into my tex code
\newcommand{\itexarray}[1]{\begin{matrix}#1\end{matrix}}

% math-mode versions of \rlap, etc
% from Alexander Perlis, "A complement to \smash, \llap, and lap"
%   http://math.arizona.edu/~aprl/publications/mathclap/
\def\clap#1{\hbox to 0pt{\hss#1\hss}}
\def\mathllap{\mathpalette\mathllapinternal}
\def\mathrlap{\mathpalette\mathrlapinternal}
\def\mathclap{\mathpalette\mathclapinternal}
\def\mathllapinternal#1#2{\llap{$\mathsurround=0pt#1{#2}$}}
\def\mathrlapinternal#1#2{\rlap{$\mathsurround=0pt#1{#2}$}}
\def\mathclapinternal#1#2{\clap{$\mathsurround=0pt#1{#2}$}}

\newcommand{\darr}{\downarrow}


% % Some conventions to start using:
% 1)capitalized things will typically not require arguments as they should be proper nouns.
% E.g. \cat{C} ought to turn C into a catgeory while $\Cat$ will merely be the category of categories
% 2) all-caps indicates the higher categorical version of something. E.g. above \CAT would give \infty-cats

%An attempt at introducing a font system for all of the different kinds of objects

\newcommand{\ot}{\leftarrow}
\newcommand{\id}{1}

%Objects

%% \newcommand{\ob}[1]{\mathcal{#1}}
\newcommand{\meta}[1]{\mathcal{#1}}
\newcommand{\Ob}{\meta{Obj}}
\newcommand{\obj}[1]{\meta{Obj}(#1)}
%1-categories 
\newcommand{\cat}[1]{\mathsf{#1}}
\newcommand{\Hom}{\meta{Hom}}
\renewcommand{\hom}[1]{\Hom_{#1}}
\newcommand{\Fun}{\cat{Fun}}
\newcommand{\Psh}{\cat{Psh}}
\newcommand{\Grpd}{\cat{Grpd}}
\newcommand{\Grp}{\cat{Grp}}
\newcommand{\psh}[1]{\Psh_{#1}}
\newcommand{\grpd}[1]{\Grpd \left\(#1 \right\)}
\newcommand{\grp}[1]{\Grp \left\(#1 \right\)}
\newcommand{\Yo}{\mathcal{Y}} %yoneda
\newcommand{\yo}[1]{\Yo_{#1}} %yoneda embedding
\renewcommand{\L}{\mathscr{L}} %localization functor
%Higher categories (caps)
\newcommand{\CAT}[1]{\mathcal{#1}}%TODO decide on higher category convention
\newcommand{\Cat}{\cat{CAT}} %TODO this was changed to fit convention 1, but may cause errors
%Internal Categories

%Objects in Internal Categories

\newcommand{\ZZ}{\mathbb{Z} }
\renewcommand{\SS}{\mathbb{S} }
\newcommand{\NN}{\mathbb{N} }
\newcommand{\RR}{\mathbb{R} }
\newcommand{\QQ}{\mathbb{Q} }


\newcommand{\G}{\mathbb{G} }
% \newcommand{\H}{\mathbb{H} } %conflicts with existing command. Not sure if worth it
\newcommand{\I}{\mathbf{I}}
\newcommand{\GG}{\mathbb{G} }
\newcommand{\B}[1]{\cat{B}\mathbb{#1}} %Classifying spaces
\newcommand{\PP}{\mathbb{P} }
\newcommand{\FF}{\mathbb{F} }
%Basic Cats
\newcommand{\CC}{\mathbb{C} }
\newcommand{\DD}{\mathbb{D} }
\newcommand{\internal}[1]{\mathbb{#1}}
\renewcommand{\P}{\internal{P} } %TODO need new convention for internal objects
\newcommand{\C}{\cat{C} }
\newcommand{\D}{\cat{D} }
\newcommand{\induced}[1]{\widehat{\left( #1 \right)}}
\newcommand{\grpd}[1]{\underline{#1}}
%\newcommand{\J}{\mathsf{J} }
\newcommand{\terminal}{*}
\newcommand{\cod}[1]{cod\left( #1 \right)}
\newcommand{\comp}{\circ }
\newcommand{\inj}{\hookrightarrow }
\newcommand{\surj}{\twoheadrightarrow }
\newcommand{\nat}{\Rightarrow }
%% \newcommand{\cat}[1]{\mathrm{#1} }
%% \newcommand{\CAT}{\cat{CAT} }
%Topology Stuff
\newcommand{\Top}{\cat{Top} }
\newcommand{\Open}{\mathscr{O} }
%Topos Stuff
\newcommand{\topos}[1]{\ensuremath{\mathcal{#1}}}
\newcommand{\topology}[1]{\ensuremath{\underline{#1}}}
\newcommand{\tJ}{\topology{J}}
\newcommand{\tK}{\topology{K}}
\newcommand{\siteC}{\left( \C, \tJ \right)}
\newcommand{\siteD}{\left( \D, \tK \right)}
\newcommand{\Sh}{\cat{Sh}}
\newcommand{\sheafification}[1]{\mathbf{a}\left( #1 \right)}
\newcommand{\E}{\topos{E} }
\newcommand{\F}{\topos{F} }
%Higher Cats/ Internal Cats
\newcommand{\iCAT}{\cat{CAT}_\infty }
\newcommand{\iGpd}{\cat{Gpd}_\infty }
\newcommand{\iGr}{\cat{Groth}_\infty }
\newcommand{\Gr}{\cat{Groth} }
\newcommand{\Loc}{\cat{Loc} }
\newcommand{\Simp}{\cat{\Delta} }
\newcommand{\Sets}{\cat{Set} }
\newcommand{\Set}{\cat{Set} }
\newcommand{\sSet}{\cat{sSet} }
\newcommand{\sSets}{\cat{sSet} }
\newcommand{\sS}{\cat{sS} }
\newcommand{\s}[1]{\cat{s #1} }
\newcommand{\Horn}{\Lambda }
\newcommand{\del}{\partial}
\newcommand{\isisom}{\cong}
\newcommand{\iso}{\cong}
\newcommand{\tensor}{\otimes}
% \renewcommand{\equiv}{\simeq} %conflicts with mathabx

\renewcommand{\Vec}[1]{\cat{Vec}_{#1}}
\newcommand{\Grp}{\cat{Grp}}
\newcommand{\Ab}{\cat{Ab}}
\newcommand{\Man}{\cat{Man}}
\newcommand{\Mod}[1]{\cat{Mod}_{#1}}

\newcommand{\completion}[1]{\overline{#1}}
\newcommand{\liminj}{\varinjlim}
\newcommand{\colim}{\varinjlim}
\newcommand{\limproj}{\varprojlim}
\renewcommand{\lim}{\varprojlim}


\newcommand{\Gpd}{\cat{Gpd} }
\newcommand{\Aut}{\mathsf{Aut} }
\newcommand{\act}{\circlearrowright}

\newcommand\restr[2]{{% we make the whole thing an ordinary symbol
  \left.\kern-\nulldelimiterspace % automatically resize the bar with \right
  #1 % the function
  \vphantom{\big|} % pretend it's a little taller at normal size
  \right|_{#2} % this is the delimiter
  }}

%DIAGRAM SHORTCUTS
%to be used in tikzcd environment
\newcommand{\internalGroupoid}[3]{
  #1  \arrow[r,  shift right=2] \arrow[r] \arrow[r,  shift left] & #2 \arrow[l, shift left=1]  \arrow[l, shift right=1]   \arrow[r, shift left=1] \arrow[r, shift right=1] & #3 \arrow[l] \\
}


\begin{document}
%================================================================================
\section{1-cat stuff}
%================================================================================
\subsection{fibrations}
%================================================================================
(reference is Borceux Vol. 2 \cite{borceuxV2})
\begin{definition}
\label{cartesian-map}
Given a functor of $1$-categories $F: \E\to \B$, and a map $\alpha : J \to I$ in $\B$ we say $f:Y\to X$ is \de{ $F$-cartesian (over $\alpha$)} if
\begin{enumerate}
\item $F(f)=\alpha$
  \item for every $g: Z\to X$ if for some $\beta : K \to J$ we have $F(g)= \alpha\comp \beta$, then there is a unique $H:Z\to Y$ s.t $F(h)\beta$ and $g = f\comp h$
\end{enumerate}
If context is obvious, we will simply say $f$ is cartesian.
\end{definition}
In other words, $f$ is an $F$-lift of $\alpha$ and any other map into the codomain of $f$ which factors in \B (upon applying $F$) also factors uniquely in $\E$.
\begin{definition}
  A functor $F:\E \to \B$ is a \de{cartesian fibration} if for every $\alpha: \J \to I$ in $\B$ and every $x\in \E$ s.t. $F(x)= I$, there exists an $f:Y\to X$ which is $F$-cartesian over $\alpha$.
\end{definition}
\begin{xample}
Let $\C$ be a category (not necessarily small), and consider the category $\Set(\C)$ whose objects are sets $I$ with an $I$-indexed collection $ \{ C_i\}_{i\in I} $ of objects in $\C$.
Morphisms of $\Set(\C)$ are morphisms of index sets $p:I \to J$ along with an $I$-indexed collection of maps $\{f_i:C_i \to D_{p(i)}\}_{i\in I}$.
The projection functor $\pi: \Set (\C) \to \Set$ (sending $(I,\{C_i\})$ to $I$) is a cartesian functor.
%TODO path lifting intuition
%TODO what are the cartesian lifts
\end{xample}

\begin{xample}
  Given a category $\C$ we have the codomain functor $\C^\to \to \C$. This is a fibration.
  (say more?)
\end{xample}
Given the above example, we can highlight another approach to thinking about cartesian fibrations, namely that they are analogous ot maps of spaces which satisify the homotopy lifting (or is it extension?) property.
However, ``paths'' and ``homotopies'' are directed here, so we have to specify whether we'd like to extend from the target or from the source (be more precise here).
The following definitions make this approach more explicit.

\begin{definition}
In the notational context of definition~\ref{def:cartesian-map}, we say $f$ is \de{pre-cartesian} if $f$ is the final amongst lifts of $\alpha$ (with the same target as $f$).
\end{definition}
Note that if we already know $F$ is cartesian, then pre-cartesian and cartesian morphisms are the same.
\begin{definition}
  In the notational context of definition~\ref{def:cartesian-map}, we say $F$ is a \de{cartesian fibration} if
  \begin{enumerate}
    \item every $\alpha: J \to I$ and $x\in F^{-1}(I)$ has some pre-cartesian lift
      \item pre-cartesian lifts are closed under composition
  \end{enumerate}
\end{definition}


%================================================================================
\subsection{internal locales}
%================================================================================
\subsection{internal locales}
%================================================================================
%================================================================================
%================================================================================
%================================================================================


\section{Problem Statement}
\subsection{internal locales}
\subsection{cartesian fibrations}
\begin{definition}(cite Aaron's definition)
  An $\infty$-functor $\ifun{F}:\icat{E} \to \icat{B}$ is a \de{cartesian fibration} if \ldots
\end{definition}
  \section{CSS}

\end{document}
